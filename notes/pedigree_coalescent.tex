
%%% Local Variables: 
%%% mode: latex
%%% TeX-master: t
%%% End: 

\documentclass{article}
%\documentclass[aps,prl,showpacs,twocolumn]{revtex4}
\usepackage{graphicx}
\usepackage{amsmath}
\usepackage{natbib}
\newcommand{\bm}[1]{\mbox{\boldmath $#1$}}
\newcommand{\EQ}[1]{Eq.~(\ref{eq:#1})}
\newcommand{\EQS}[2]{Eqs.~(\ref{eq:#1)} and (\ref{eq:#2})}
\newcommand{\FIG}[1]{Fig.~\ref{fig:#1}}
\newcommand{\REF}[1]{ref.~\citep{#1}}

\begin{document}
\title{Contig Coalescent}
\maketitle
This is an attempt to redo Boris' calculation and free it from
historical baggage. We will be considering biparental reproduction
with exactly one crossover event per chromosome that falls into the
interval $[0,1]$. Individuals are numbered from $1\ldots N$ and we
denote the parents of individual $j$ by $m_j$ and $p_j$. For a
particular location $x$ on the genome, the index of the ancestor is
therefore 
\begin{equation}
  a(j,x) = \begin{cases}
    m_j & x< x_j \\
    p_j & x\geq x_j
  \end{cases}
\end{equation}
The contributions of the ancestral generation to the present
generation can thus be encoded in a sparsely filled matrix
\begin{equation}
  M_{ij}(x)=\delta_{i,a(j,x)}
\end{equation}
Likewise, the contributions of generation $T-2$ to generation $T$ is given by
\begin{equation}
  M_{ik}(x) M_{kj}(x) = \sum_k \delta_{i,a(k,x)}\delta_{k,a(j,x)}
\end{equation}
more generally, we can define the weight distribution $w_i(T,x)$ of
individual $i$ to the present generation at locus $x$. It obeys
\begin{equation}
  w_i(x,T) = (M\times M\times \ldots M)_{ij}w_j(x,0) = g_{ij}(x,T)w_j(x,0)
\end{equation}
where $w_j(x,0)=1$. 
Obviously, we have
\begin{equation}
  \sum_i g_{ij}=1 
\end{equation}
which is to say that each locus has got to come from somewhere. Also
note that the $g_{ij}(x,T)$ obey
\begin{equation}
  g_{ij}(x,T)g_{kj}(x,T) = \delta_{ik}g_{kj}(x,T) 
\end{equation}
This is true, since only one individual can contribute to locus $x$ at
time $T$, hence $\delta_{ik}$, and $g_{ij}\in \{0,1\}$.

\section*{Coalescence}
Consider the question whether two loci $x$ and $x'$ of two individuals $i$ and
$j$ derive from the same ancestor. This is simply
\begin{equation}
  C_{ij}(x,x'|T) = \sum_k g_{ki}(x,T)g_{kj}(x',T) = \left[\prod \mathbf{M}(x,t)\right]^{\dagger}\left[\prod \mathbf{M}(x',t)\right]
\end{equation}
$C_{ij}(x,x'|T)$ takes values $0,1$, the sum is over possible
ancestors and the matrices are quenched realizations. We can break the
product into generations $0,\ldots T-1$ and the last:
\begin{equation}
  \begin{split}
  C_{ij}(x,x'|T) = &\sum_{k,n,m} M_{km}(x,T)
  g_{mi}(x,T-1)g_{nj}(x',T-1)M_{kn}(x',T)\\ 
  = &\sum_{k,m} M_{km}(x,T) g_{mi}(x,T-1)g_{mj}(x',T-1)M_{km}(x',T)\\
  &+ \sum_{k,n\neq
    m}M_{km}(x,T)g_{mi}(x,T-1)g_{nj}(x',T-1)M_{kn}(x',T)\\
= &\sum_{k,m} \delta_{k, a(m,x)}\delta_{k, a(m,x')} g_{mi}(x,T-1)g_{mj}(x',T-1)\\
  &+ \sum_{k,n\neq
    m}\delta_{k, a(m,x)} \delta_{k, a(n,x')} g_{mi}(x,T-1)g_{nj}(x',T-1)
  \end{split}
\end{equation}
If we now average over different pedigrees and recombination
breakpoints, we obtain an equation for the probability
$p_{ij}(x,x'|T)$ that locus $x$ in individual $i$ and $x'$ in
individual $j$ derive from the same ancestor $T$ generations in the
past. For the first term, we do the sum over $k$. One needs to decide
whether we allow selfing. We can probably ignore it, but I keep it for
now. 
\begin{equation}
  \begin{split}
   p_{ij}(x,x'|T) =& (1-|x-x'| + \frac{|x-x'|}{N})p_{ij}(x,x'|T-1)\\
   & + \frac{1}{N}\sum_{n} g_{mi}(x,T-1)\sum_{m} g_{nj}(x',T-1) -
  \frac{1}{N}p_{ij}(x,x'|T-1)\\
=& p_{ij}(x,x'|T-1) -(N^{-1} - |x-x'|(1-N^{-1}) p_{ij}(x,x'|T-1) +N^{-1}
  \end{split}
\end{equation}
The sums $\sum_{m} g_{nj}(x',T-1)=1$, while the factor $N^{-1}$ comes
from the probability that $a(m,x)=a(n,x)$ which in general depends on
the offspring variance (assumed 1 here). Hence we have
\begin{equation}
   \frac{d}{dT} p_{ij}(x,x'|T) = -(N^{-1} - |x-x'|(1-N^{-1}) p_{ij}(x,x'|T-1) +N^{-1}
\end{equation}
It is convienient to define $q=1-p$ as the probability of not having a
common ancestor, which evolves according to
\begin{equation}
   \frac{d}{dT} q_{ij}(x,x'|T) = -N^{-1}q_{ij}(x,x'|T)+
   |x-x'|(1-N^{-1}) p_{ij}(x,x'|T-1)
\end{equation}
This reflects the opposing tendencies of coalescence and
recombination. If $x=x'$, we recover the simple coalescent with rate
$N^{-1}$. The $N^{-1}$ correction is due to selfing and can be
dropped. 

Next, let's look at the event of the ancestor of at loci $x_i$, $x_j$,
$x_K$ in individuals $i,j,k$ being identical $T$ generations in the
past. 
\begin{equation}
  \begin{split}
    C(x_i,x_j,x_k |T) =& \sum_n g_{ni}(x_i,T)g_{nj}(x_j,T)g_{nk}(x_k,T) \\
    =&
    \sum_{n,rst}\delta_{n,a(r,x_i)}\delta_{n,a(s,x_j)}\delta_{n,a(t,x_k)}
    g_{ri}(x_i,T)g_{sj}(x_j,T)g_{tk}(x_k,T)  
  \end{split}
\end{equation}
This can now be decomposed into the different cases where $r=s=t$,
two of $r,s,t$ are equal, or non of them. For simplicity, we will
ignore subleading terms resulting from selfing right away. Averaging
yields
\begin{equation}
  \label{eq:triple}
  \begin{split}
    p(x_i,x_j,x_k |T) =& (1-|x_i-x_k|-3N^{-1}) p(x_i,x_j,x_k |T-1) \\&
    + N^{-1}(1-|x_i-x_j|)p(x_j,x_k|T-1) 
    + N^{-1}(1-|x_i-x_k|)p(x_i,x_k|T-1) \\&
    + N^{-1}(1-|x_j-x_k|)p(x_j,x_k|T-1) + N^{-2}
  \end{split}
\end{equation}
The last term accounts for a triplett coalescent event, which is
suppressed by an extra factor of $N$ and can be ignored, at least in
the neutral case. The $-3N^{-1}$ is again the correction for the
double counting that happened when summing over the cases $r\neq s=t$
but extending the sum to all $r$.
Denoting the maximal distance between a $x_i,\dots,x_k$ times $N$
by $\xi_{i\ldots k}$, rescaling time $\tau = T/N$, and dropping
subleading terms, we find
\begin{equation}
  \begin{split}
   \frac{d}{d\tau} p(x_i,x_j,x_k |\tau) =& -(3+\xi_{ijk})p(x_i,x_j,x_k |\tau) 
    +p(x_j,x_k|\tau) 
    +p(x_i,x_k|\tau) 
    +p(x_j,x_k|\tau) 
  \end{split}
\end{equation}
Similar decompositions can be done for higher order terms, see Boris'
note.

\section*{Selection}
In the previous calculation, we have used the fact that the offspring
distribution is the same for every individual and have assumed that
it's variance is one. In addition, there has been the implicit
constraint that the mean offspring number equals 1.

To model selection, individuals need an additional attribute
``fitness'' that determines their offspring number. We also need to
maintain a constraint on the overall population size. Individuals can
coalesce only if the are close enough in fitness that they can
plausibly have descented from the same ancestor. More precisely, we
can define a forward propagator $q(f_i|f_j)$ to change fitness from
$f_j\to f_j$. Let's consider asexual coalescence first. The
probability that two individuals $i,j$ have the same parent is
\begin{equation}
  p(f_i,f_j) = \sum_k q(f_i|f_k)q(f_j|f_k)
\end{equation}
\end{document}
