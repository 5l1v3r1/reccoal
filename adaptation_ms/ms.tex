\begin{abstract}
\noindent
In diverse sexual populations, selection operates neither on the whole
genome -- which is repeatedly taken apart and reassembled by
recombination -- nor on individual alleles which are tightly linked to
the chromosomal neighborhood. Those tightly linked alleles affect each
others dynamics which reduces the efficiency of selection and distorts
patterns of genetic diversity. Inference of evolutionary history from
diversity shaped by linked selection requires an understanding of these
patterns.  Here, we reexamine this problem in the light of recent
progress in coalescent theory of rapidly adapting asexual
populations. We present a simple but powerful scaling analysis
identifying the unit of selection as the genomic ``linkage block'' with
characteristic length $\blockT$, which is determined in a
self-consistent manner by the condition that the rate of recombination
within the block is comparable to the fitness differences between
different alleles of the block.  We find that an asexual model with
strength of selection tuned to that of the linkage block provides an
excellent description of genetic diversity and the site frequency
spectra when compared to computer simulations of population
dynamics. This correspondence holds for the entire spectrum of strength
of selection. When fitness differentials arise from the
collective contribution of numerous weakly selected polymorphisms, the
rate of adaptation increases as the square root of the recombination
rate. Linkage block approximation thus provides a simple but powerful
tool for understanding interference and collective behavior of dense
weakly selected loci.
\end{abstract}
%%%%%%%%%%%%%%%%%%%%%%%%%%%%%%%%%%%%%%%%%%%%%%%%%%%%%%%%%%%%%%%%%%%%%%%%%

%%%%%%%%%%%%%%%%%%%%%%%%%%%%%%%%%%%%%%%%%%%%%%%%%%%%%%%%%%%%%%%%%%%%%%%%%
%\section{Introduction}
%%%%%%%%%%%%%%%%%%%%%%%%%%%%%%%%%%%%%%%%%%%%%%%%%%%%%%%%%%%%%%%%%%%%%%%%%
In asexual populations different genomes compete for survival and the
fate of most new mutations depends more on the total fitness of the genome they
resides in than on their own contribution to fitness.  In this way,
different genetic loci on the same genome interact or ``interfere'' with
each other. As a result, beneficial mutations can be lost to competition
with other clones, an effect known as ``clonal interference''
\cite{Gerrish:1998p5933,Desai:2007p954,neher_genetic_2013}, while
deleterious mutations in very fit genomes can fix. This inefficiency of
selection is reduced by sexual reproduction and disappears in an idealized limit
where all loci are reassorted fast enough so
that selection can act independently on different loci. 

In most organisms, recombination of genetic material is achieved by
crossing over of homologous chromosomes. As a result, the fates of distant loci are
rapidly decoupled, but the interference between
nearby tightly linked loci remains. This effect of linked selection, known as
Hill-Robertson interference \cite{Hill:1966p21029} or
``genetic draft'', is particularly important in diverse populations
where polymorphisms are dense \cite{cutter_molecular_2013} and in
faculatively outcrossing species where the effective rate of recombination is
low
\cite{bomblies_local-scale_2010,tsai_population_2008,barriere_high_2005}.

One indication of the importance of linked selection is
the weak correlation between levels of genetic diversity -- a measure of
the typical time to the most recent common ancestor at a particular
locus -- and the population size
\cite{leffler_revisiting_2012,Lewontin1974}. In conflict with
observations, classic neutral models of population genetics predict that
diversity should increase linearly with the population size
\cite{Kingman:1982p28911,Kimura:1964p3388}.  This ``paradox of
variation'' is resolved if the time to the most recent common ancestor
(MRCA) is controlled by selection rather than random genetic drift
\cite{Smith:1974p34217,Gillespie:2001p9636}. 
If linked selection
rather than drift drives coalescence towards the MRCA, a weak
correlation between population size and diversity is expected.


% In addition to reducing genetic diversity, linked selection also reduced
% the efficacy of selection. This effect was first described by Hill and
% Robertson \cite{Hill:1966p21029}, who observed that 
% the fixation probability of two beneficial mutations increases with the
% recombination rate between those loci. \cite{Barton:1994p34628} has shown that a sweeping allele with
% selection coefficient $s$ reduces the fixation probability of weaker
% alleles over a time $\approx s^{-1}$ in a segment of the chromosome of
% width $s/\xo$, where $\xo$ is the crossover rate.
% Weissman and Barton \cite{weissman_limits_2012} have extended this argument to the joint
% effect of many sweeping loci and shown that the rate of adaptation is
% primarily limited by the length of the genetic map.  In addition to
% beneficial mutations, linkage to deleterious mutations reduces fixation
% probabilities and genetic diversity
% \cite{Charlesworth:1993p36005}. Background selection models have been
% generalized to recombining populations
% \cite{Hudson:1995p18197,Nordborg:1996p18149}, in which it has been shown that
% the reduction of genetic diversity depends on the ratio of the
% deleterious mutation rate and the crossover rate.

Many populations harbor substantial heritable phenotypic variation which
in a largely unknown way depends on a large number of polymorphisms in
the genome. The majority of these polymorphisms have a
small effects on phenotypes and fitness. Collectively, however, they can
dominate phenotypic variation \cite{Yang:2010p35988} and possibly
selection. Genetic variance collectively generated by weak
(with and perhaps individually undetectable effects) alleles may be
aptly thought of as ``genomic dark matter'' -- a term that we shall use
below when referring to that limit. The dynamics of evolution and
patterns of diversity in populations dominated by such dark matter are
largely uncharacterized.
Existing work has mainly assumed that selection operates
on a small number of strongly selected loci whose dynamics is
deterministic, at least at high frequencies
\cite{Hill:1966p21029,Barton:1994p34628,weissman_limits_2012}; we discuss
the relation of this work to ours below. 

Here, we characterize the structure of genealogies, genetic diversity
and the rate of adaptation in sexual populations in the genomic dark matter limit.
We build on recent progress in the understanding of genealogies in
adapting asexual populations
\cite{neher_genealogies_2012,desai_genetic_2012,Brunet:2007p18866},
which deviate strongly from the Kingman's coalescent
\cite{Kingman:1982p28911} characteristic of neutral populations. We
shall start by reviewing these results and their implication for
observables such as the the site frequency spectra (SFS) of derived
alleles
\cite{Braverman:1995p34932,Neher:2011p42539,neher_genealogies_2012}. We
shall than present a scaling argument that reduces the coalescence
within an adapting sexually reproducing population to the already solved
problem of asexual coalescence driven by selection and allows us to
predict levels of genetic diversity, coalescent time scales, and site
frequency spectra (SFS). We confirm the validity of the mapping to the asexual
model by comparing its predictions with numerical simulations of
evolving sexual populations. We use this approximation to
demonstrate that in the ``genomic dark matter'' limit the rate of
adaptation scales as a square root of recombination rate: a result also
confirmed by numerical simulations.
%In large and rapidly adapting populations, the time scale of coalescence is almost completely decoupled from the population size but depends primarily on the variance in fitness. In addition to reducing diversity, selection also changes the properties of coalescence trees, which for example results in the site frequency spectra (SFS) of derived alleles being non-monotonic with an excess of rare and common alleles \cite{Braverman:1995p34932,Neher:2011p42539,neher_genealogies_2012}. We show that genetic diversity in adapting sexual populations is well described by asexual models restricted to distances over which loci are in linkage disequilibrium.



\section{Results}
In a sexual population, different loci on the same chromosome can have
different genealogical trees since they can be inherited from different
(grand) parents. Neighboring loci, however, are separated rarely and
locally the ancestral recombination graph is approximately asexual.
Coalescence and interference is well understood in asexual populations,
and we will use these results to shed light on the genetic diversity in
adapting sexual populations. To this end, we determine the maximal
segment length that coalesces into a common ancestor before it is
disassembled by recombination. This yields self-consistent predictions
for genetic diversity and the SFS that compare very well to
simulations. In order to make the paper self-contained, we will quickly
summarize the results on coalescence in asexual populations before
discussing sexual populations.

\subsection*{Coalescence in adapting asexual populations.}
If fitness depends on a large number of polymorphic loci
with comparable effects, the fitness distribution in the population will
be roughly Gaussian, and the fittest individuals are $x_c \approx \sigma
\sqrt{2\log N\sigma}$ ahead of the fitness mean, where $\sigma$ is the
total fitness variance in the
population\cite{Tsimring:1996p19688,Rouzine:2003p33590,Desai:2007p954}. 
In large asexual populations, only individuals in the high
fitness nose have an appreciable chance to contribute to future generations.  
It will take those individuals roughly $\sigma^{-1}\sqrt{\log N\sigma}$
generations to dominate the population, which sets the time scale of
coalescence.  In contrast, in small populations with $N\sigma \ll 1$,
coalescence is dominated by neutral processes (non-heritable
fluctuations in offspring number known as genetic drift). The average 
number of generations back to the most recent common ancestor of any
pair of extant genomes, a.k.a. the pair coalescence time, is given by:
\begin{equation}
  \label{eq:asex}
  \mTtwo \approx
  \begin{cases}
    N & N\sigma \ll 1 \\
    c\sigma^{-1}\sqrt{2\log N\sigma} & N\sigma\gg 1
  \end{cases}
\end{equation}
where $c$ is a number of order one
\cite{neher_genealogies_2012,desai_genetic_2012}. In an attempt to
extend applicability of the neutral coalescent, one sometimes defines an
``effective population size'', $\Ne$, equal to $\mTtwo$ regardless of whether
coalescence is neutral or not. By definition a neutral model with
$\Ne=\mTtwo$ predicts the same levels of genetic diversity, but the
statistical properties of the genealogies dominated by selection are quite
different and cannot be papered over simply by redefining the population
size. We will therefore avoid the term $\Ne$ and stick to $\mTtwo$. 

For the approximately neutral case, $N\sigma\ll
1$, the coalescent tree is of the Kingman type
\cite{Kingman:1982p28911}, while for $N\sigma \gg 1$ it resembles the
Bolthausen-Sznitman coalescent (BSC)
\cite{Bolthausen:1998p47390,Brunet:2007p18866}. Two representative
coalescent trees sampled from asexual populations, one neutral and one
rapidly adapting are shown in \FigSketch A.  

%The Kingman coalescent and the
%BSC correspond to qualitatively
%different population genetics: the Kingman coalescent emerges when
%frequencies of neutral alleles change through drift, while the BSC
%corresponds dynamics dominated by selection.
%genetic draft, i.e., fluctuations that result from heritable associations with genetic backgrounds of different fitness \cite{neher_genetic_2013}.

\begin{figure}[th]
  \centering
  \includegraphics[width=0.98\columnwidth]{sketch}
  \caption{
    Coalescence in neutral and adapting populations. Panel (A) shows a
    typical coalescent tree from neutral and adapting asexual
    populations (left and right, respectively). In adapting populations,
    coalescent trees branch asymmetrically and contain
    approximate multiple mergers. Panel (B) illustrates 
    asexual blocks in sexual populations. The sketch depicts a
    representative chromosome at the bottom with polymorphisms
    indicated as balls. Different loci within segments shorter than
    $\blockT$ share  most of their genealogical history, i.e., have
    trees similar to the one indicated in the center of the
    segment. Coalescence within this segment of length $\blockT$ is
    either neutral or driven by the fitness differences between
    different haplotypes spanning these segments. 
    The fitness distribution of these haplotype blocks is indicated as
    inset. Distant parts of the chromosome are in linkage
    equilibrium and the trees changes as one
    moves along the chromosome. The succession of changing trees
    is the ancestral recombination graph.  
  }
  \label{fig:sketch}
\end{figure}

\subsection*{Sexual populations and recombination.}
While in the asexual case all loci share the same genealogy and
are selected as one unit, recombination decouples different loci -- the
further apart, the more rapidly. The typical length of the segment that
is not interrupted over a time $t$ along one ancestral lineage decreases
with time as
\begin{equation}
  \label{eq:block_length}
  \block = \frac{L}{1+L\xo t} \approx \frac{1}{\xo t}
\end{equation}
where $\xo$ is the crossover rate and $L$ is the length of the chromosome.
The second approximation is justified whenever $\block\ll L$. If polymorphisms
affecting fitness are spread evenly across the genome, we expect that
a different segregating haplotypes with a segment of length $\block(t)$ 
harbor fitness variation proportional to its length 
\begin{equation}
  \label{eq:sigma_block}
  \sigblock^2 = \frac{\block(t)}{L}\sigma^2 \ .
\end{equation}
This fitness variance shrinks as the block length decreases over
time. While initial fitness differences between blocks are large, they
are chopped into smaller blocks so rapidly that selection has no time to
amplify the fittest of these early large blocks. But the rate at which
blocks are chopped down decreases as they get shorter, and at some point
the rate of chopping them down is outweighed by the amplification of the
fittest blocks by selection. The latter happens when fitness differences
between haplotypes of this block are comparable to the recombination rate.
More precisely, the relevant block length $\block(t)$ is the
length that survives over the time scale of coalescence, i.e.,
$\blockT=\block(\mTtwo)$. In large enough populations, the time scale of
coalescence itself is determined by these fitness differences via
\EQ{asex}, only that in sexual populations the 
fitness variance, $\sigblockT^2$,  within the linkage block of length $\blockT$ sets the
relevant scale of selection rather than the total variance $\sigma^2$
as in asexual populations.  Using $\mTtwo = c{\sigblockT}^{-1}\sqrt{2\log
  N\sigblockT}$ in \EQ{block_length}, we find for the length of linked
blocks 
\begin{equation}
  \label{eq:blockTc}
  \blockT = \frac{\sigblockT}{c\xo \sqrt{2\log N\sigblockT}} \ .
\end{equation}
Substituting $\blockT$ into \EQ{sigma_block}, yields
\begin{equation}
  \label{eq:self_consist}
  \sigblockT = \frac{\sigma^2}{L\xo c\sqrt{2\log N\sigblockT}} \quad
  \mathrm{and} \quad \blockT = \frac{\sigma^2}{2Lc\xo^2 \log
    N\sigblockT} \ .
\end{equation}
Hence the relevant fitness scale is given by the
fitness variance per maplength $r = L\rho$ with a logarithmic correction. 
To arrive at this result, we have assumed that coalescence is driven
by selection, i.e., we have assumed $N\sigblockT\gg 1$. If this
condition is not satified, local coalescence will be approximately
neutral. In this case $\mTtwo = N$ and the LD extends over
$\blockT\sim (N\rho)^{-1}$ nucleotides.

The condition for draft dominance, $N\sigblockT\gg 1$, is more
stringent in sexual populations than in asexual populations,
$N\sigma\gg 1$. In other words, recombination reduces interference and
results in drift dominated coalescence over a larger parameter range. 
Our prediction now is that the results for genetic diversity in the
asexual coalescent apply with $\sigblockT^2$ as the local fitness variance,
and that linkage disequilibrium between common loci extends over
a distance $\blockT$. We will validate these predictions by forward
simulations of different population models.

\subsection*{Fixed rate of adaptation.}
The first model we consider is a population in which the fitness
variance is set by external (environmental) factors; see Model and
Methods. This model might be a first approximation to scenarios where
selection pressures are dictated by a changing environment, an evolving
immune system, or a breeder who imposes a certain selection differential
during artificial selection.  We simulate our populations using a
discrete generation model with an approximately constant population size
and a finite number of sites in the genome as implemented in FFPopSim
\cite{zanini_ffpopsim:_2012} (see Methods).  In addition to simple
forward simulation, we track the genealogy of a locus in the center of
the chromosome, which allows us to study properties of representative
coalescent trees such as pair coalescence times or the SFS of derived
alleles (SFS).

After equilibration, we sample the evolving population in roughly
$\mTtwo$ intervals and measure $T_2$, $T_{MRCA}$, the SFS, and the
linkage disequilibrium (LD) between polymorphisms at intermediate
frequencies ($[0.1, 0.9]$). We perform these simulations for many
combinations of parameters. For each of these combinations, we calculate
$\sigblockT$ according to \EQ{self_consist}. \FigTtwo~shows
that the average pair coalescence time $\mTtwo$ approaches $N$ for
$N\sigblockT\to 0$ and that it is proportional to $\sigblockT^{-1}$ (with
logarithmic corrections) for $N\sigblockT\gg 1$ as prediced.

\begin{figure}
  \centering
  \includegraphics[width=0.91\columnwidth]{T2}
  \caption{Coalesence in sexual populations. The figure shows the
    average pair
    coalescent time $\mTtwo$ relative to the neutral expectation as a
    function of $N\sigblockT$ determined using \EQ{self_consist}. For $N\sigblockT\ll 1$, $\mTtwo\approx N$,
    while $\mTtwo\sim \sigblockT^{-1}$ otherwise. }
  \label{fig:Ttwo}
\end{figure}

In addition to the reduction in genetic diversity, we predict that the
local genealogies will resemble the BSC
rather than the Kingman coalescent whenever $N\sigblockT\gg 1$. One
characteristic of the BSC is a non-monotonic SFS with a steep
divergence $f(\nu)\sim\nu^{-2}$ for rare alleles. \FigSFS~shows a
collection of SFS colored by the parameter $N\sigblockT$. For large
$N\sigblockT$, the SFS agree with the prediction of the BSC and
smoothly interpolate between the BSC result and the Kingman coalescent
as $N\sigblockT$ decreases. As soon as the SFS starts deviating from
the prediction of the Kingman coalescent, Tajima's D and related
measures turn negative.  


\begin{figure}
  \centering
  \includegraphics[width=0.91\columnwidth]{SFS}  
  \caption{Site frequency spectra (SFS). The figure shows the SFS for a
    large number of simulations. Color indicates the value of
    $N\sigblockT$. For large $N\sigblockT$, the SFS display the
    non-monotonicity characteristic of the BSC (dashed line), while the
    SFS are well described by the prediction from Kingman's coalescent
    (solid line) if $N\sigblockT\ll 1$. }
  \label{fig:SFS}
\end{figure}

The SFS records properties of the genealogy at a particular
locus. Another important feature of diversity in sexual populations is
the genomic distance across which loci share much of their
genealogy. This can be quantified by measuring the correlations between
loci (LD) at different distances. In order for our picture to be
consistent, the extend of LD should be approximately equal to $\blockT = (\xo
\mTtwo)^{-1}$. We measured LD as $r^2(d)$ for different distances $d$ and
plot its distance dependence against $d/\blockT$; see \FigLD. As
predicted, the distance over which loci are correlated
is well described by $\blockT =(\xo T_2)^{-1}$.

\begin{figure}
  \centering
  \includegraphics[width=0.91\columnwidth]{LD}
  \caption{Correlation length along the genome. The figure shows
    linkage disequilibrium, quantified as average $r^2$, at different
    distances (the curves are normalized to their value at zero
    distance). The x-axis shows the distance between loci $d$ rescaled by
    $\blockT$ determined using \EQ{block_length} with $t$ equal to the measured pair
    coalescence time. After this rescaling, the distance dependence of all
    simulations follow approximately the same master curve, which
    shows that LD extends for $\approx \blockT$. }
  \label{fig:LD}
\end{figure}


In our second model, we keep mutational effects constant instead of
adjusting them to keep the overall fitness variance fixed. In this case,
the total fitness variance is a fluctuating quantity, but the behavior
is equivalent to the first model with the fixed variance replaced by the
average.  Figures analogous to \FigTtwo~to \FigLD~are
given in the supplement.


\subsection*{Adaptation on genomic ``dark matter''.}
We now consider a model where the fitness variance and diversity are set
by a balance between frequent novel mutations of small effect and the removal of variation by
selection, i.e., fixation of beneficial alleles and purging of
deleterious alleles. This type of model has been studied for asexual populations
\cite{Kessler:1997p19690,Cohen:2005p45154} where the total
fitness variation is given by
\begin{equation}
  \label{eq:sigma_total_diffusive}
  \sigma^2 \approx \frac{L \mu \langle s^2 \rangle}{2} \mTtwo \ .
\end{equation}
Here, $\mu$ is the mutation rate and $\langle s^2 \rangle$ is the second
moment of distribution of mutational effects
\cite{neher_genealogies_2012}. The quantity $D=\frac{\mu \langle s^2
  \rangle}{2}$ is the ``fitness diffusion'' constant of lineages in the
absence of selection, and variation in this population has accumulated
over the time it takes a few lineages to dominate the population, which is
approximately given by the half the pair coalesence
time\cite{neher_genealogies_2012}. Note that in this case $\sigma^2$
depends on  the local variance $\sigblockT^2$ through $\mTtwo$. Hence
\EQ{self_consist} is not yet explicit.
Substituting \EQ{sigma_total_diffusive} and \EQ{block_length} into
\EQ{sigma_block}, we find
\begin{equation}
  \label{eq:sigma_block_DM}
  \sigblockT^2 =  {\frac{\mu\langle s^2 \rangle}{2\xo}}
\end{equation}
Remarkably, this variance of the effectively asexual blocks is simply
the ratio of variance injection per nucleotide, $\mu \langle s^2 \rangle$, and the
crossover rate (at least while $N\sigblockT\gg 1$). The coalescence time
cancels! We therefore find for $\mTtwo$ 
\begin{equation}
  \label{eq:Tc_diffusive}
  \mTtwo \approx
  \begin{cases}
    N & N\sqrt{\mu\langle s^2 \rangle\xo^{-1}} \ll 1 \\
   c\sqrt{\frac{\xo\log N\sigblockT}{\mu \langle s^2\rangle} } & N\sqrt{\mu\langle s^2 \rangle\xo^{-1}} \gg 1\\ 
  \end{cases}
\end{equation}
where $c$ is again a constant of order 1. In the limit where coalesence
is driven by selection, the total rate of adaptation is therefore 
\begin{equation}
  \label{eq:adaptation_diffusive}
  \sigma^2 \approx cL\sqrt{\xo \mu \langle s^2\rangle \log N\sigblockT} \ .
\end{equation}

We simulate this limit of many mutations with small effect by mutating
loci at random at a high rate. Since the effects of individual loci are
small, deleterious alleles will fix often, and the population settles into
a dynamic equilibrium where the fixation of beneficial mutations is roughly canceled out
by that of deleterious mutations \cite{Goyal:2012p47382}. In this dynamic
balance, many weak effect mutations segregate, and the population is
constantly adapting to counter the influx of deleterious mutations.  As
above, the pair coalescence times (genetic diversity), SFS,
and LD scale are well predicted by $N\sigblockT$. The figures
corresponding to \FigTtwo~to \FigLD~look very similar (see Supplementary
Information). The prediction for the total fitness variance,
\EQ{adaptation_diffusive}, is compared to the simulation results in
\FigAda.

% Note that in this limit where adaptation
% draws on many mutations with tiny effects, the dependence of
% $\sigma^2$ on the parameters is different from the behavior found in
% \cite{weissman_limits_2012}, where $\sigma^2$ was proportional to the
% map length $R=\xo L$, or in a model of facultative outcrossing, where
% $\sigma^2$ is proportional to the square of the outcrossing rate
% \cite{Neher:2011p42539}. We shall explain the source of these
% differences in the Discussion section.

\begin{figure}
  \centering
  \includegraphics[width=0.95\columnwidth]{adaptation_dynbalance}
  \caption{The total fitness variation due to ``dark matter'' in a model
    where frequent deleterious mutations are balanced by beneficial mutations. The
    color shows the average number of crossovers per simulated
    segment. There is a residual dependence on $\xo$, which is due to
    large corrections to the asymptotic behavior.}
  \label{fig:adaptation_dynbalance}
\end{figure}


% \begin{figure}
%   \centering
%   \includegraphics[width=0.98\columnwidth]{../../WeakSelectionCoalescent/figures/T2}
%   \includegraphics[width=0.98\columnwidth]{../../WeakSelectionCoalescent/figures/SFS}
%   \includegraphics[width=0.98\columnwidth]{../../WeakSelectionCoalescent/figures/LD}
%   \caption{Coalescence with many weakly selected mutations.}
%   \label{fig:diffusive}
% \end{figure}

% \subsection*{Changing environments}
% We could consider a model where the mutation rate is high and the
% preferred state of mutations changes every so often. This would be
% akin to a viral population subject to immune predation or drug treatment
% etc. It might be easier to swallow than the fixed $\sigma^2$. 

\subsection*{Loosely linked loci.}
Our analysis has focused on the effect of fitness variation in
short effectively asexual blocks. As discussed above, the total
strength of selection $\sigma$ can be much larger than the fitness
differences within effectively asexual blocks $\sigblockT$. However,
a particular locus only remains linked to distant polymorphisms
for a short time, and the contribution of these distant loci
averages out. For our focus on the effect of closely linked loci to be
valid, the integral contribution of such loosely linked loci 
to drift and draft should be small compared to the effect of
fitness variation $\sigblockT$ within the segment.
Loosely linked loci are amenable to a
perturbative analysis known as Quasi-Linkage Equilibrium
\cite{Kimura:1965p3008,Barton:1991p2659,Neher:2011p45096}. In
Ref.~\cite{Neher:2011p45096}, it is shown that the stochastic dynamics of
the allele frequency $\nu_i$ at locus $i$ due to loosely linked loci is
described by the following Langevin equation:
\begin{equation}
  \label{eq:allele_langevin}
  \frac{d}{dt}\nu_i(t) = \nu_i(1-\nu_i) s_i +2\mu(1-2\nu_i) + \sum_{i\neq
    j}D_{ij}(t)s_j + \eta_i(t) \ ,
\end{equation}
where $D_{ij}(t)$ is the LD between loci $i$ and $j$, $s_j$ is the
fitness effect of the derived allele at locus $j$, and $\eta_i$ is
random noise with autocorrelation function $\langle
\eta_i(t)\eta_i(t')\rangle =N^{-1}\delta(t-t')$, representing genetic
drift. 
%Langevin equations of this type are analogous to diffusion
%equations but easier to handle in high dimensions
%\cite{WGardiner:2004p36981}. 
If the two loci are loosely linked, i.e., the
recombination rate $r_{ij}$ between them is much larger than selection on either
of them, $D_{ij}$  is also a fluctuating quantity. 
The autocorrelation function of $D_{ij}$ is \cite{Neher:2011p45096}
\begin{equation}
  \label{eq:LD_autocorr}
  \langle D_{ij}(t)D_{ij}(t')\rangle = \frac{\nu_i(1-\nu_i)\nu_j(1-\nu_j)e^{-r_{ij}|t-t'|}}{2Nr_{ij}} \ .
\end{equation}
Given this autocorrelation, we can now integrate over fluctuations due
to genetic drift and loosely linked selected loci to obtain a
renormalized diffusion coefficient, i.e., the reduction of the
``effective population size''. Reproducing Eq.~44 of
\cite{Neher:2011p45096}, we have
\begin{equation}
  \label{eq:Ne}
  \frac{N}{N_e} = 1+\frac{1}{2}\sum_{i\neq j} \nu_j(1-\nu_j)\frac{s_j^2}{r_{ij}^2}
\end{equation}
This result is similar to results in
\cite{Nordborg:1996p18149,Hudson:1995p18197,Santiago:1998p34629,Robertson:1961p36142}
in that it shows that the level of drift is increased by a factor that
depends on the square of the ratio of selection and linkage, averaged
over the genome. Note, however, that this only holds for distant pairs
where $r_{ij}\gg s_j$: the effect of ``draft'' can be represented by the
renormalization of ``drift'' only as long as LD between the loci decorrelates quickly
compared to coalescence and therefore remains weak.

If we now consider the integral effect of all loci further away than
$\xi$, it is always dominated by the loci at the smallest distance, so
that $N/\Ne -1 \sim ( \sigma / r )^2 ( \xi / L)^{-1} $ (obtained as
a continuum approximation to the sum in \EQ{LD_autocorr}). Hence
provided that fitness variation at distant loci is sufficiently small
(or the loci are sufficiently distant !) $\xi/L > ( \sigma / r)^2 $
their effect can be accounted for by a simple rescaling of the effective
population size \cite{weissman_limits_2012}: this is the {\it weak
  draft} regime. Note, however, that the recombination rate between
distant loci is ultimately limited by the outcrossing rate and that
distant loci can have substantial effects in facultatively sexual
populations \cite{Neher:2010p30641,weissman_limits_2012}.

The negligible effect of loosely linked loci is a consequence of two
types of averaging that are apparent in \EQ{LD_autocorr}: (i) The 
associations between these distant loci are transient and average out
over time. This manifests itself in the decay time of $r_{ij}^{-1}$ in
\EQ{LD_autocorr}. (ii) Different individuals carry different alleles at
these distant loci, and hence their fitness effect is averaged over
different descendends. As a consequence, the auto-correlation in \EQ{LD_autocorr} is
proportional to $(Nr_{ij})^{-1}$. Together, these two averages result
in the $1/r_{ij}^2$ contribution of loosely linked loci.

For the more tightly linked loci, i.e., $\xi <  \xi_*=( \sigma / r)^2 L
%= \mathbf{\pi} \langle s^2 \rangle \rho^{-2} 
$, the behavior crosses over
to the {\it strong draft} regime. This crossover length scale $\xi_*$ is 
controlled entirely by the {\it local}
quantities: recombination rate per base pair $\rho$ and the local
fitness variance density. Furthermore, $\xi_*$ is in general larger than
$\blockT$ with 
$\xi_*/\blockT \sim \log (N \sigblockT )$. This ratio corresponds to the
reduction in the block size during the span of time between local
selection effects first coming into play and the coalescence time. In
the limit of $\log (N \sigblockT ) \gg 1$ recombination events within
the $\xi_*$ block must be reckoned with, yet for more realistic population sizes,
we have shown above that focusing on the $\blockT$-sized asexual segment
captures the effects of strong draft quite well.

\subsection*{Length distribution of segments identital by descent (IBD).}
The structure of genealogies has implications for the length $\ell$ of
IBD segments in pairs of individuals. Their
distribution, $p(\ell )$, is directly related to the distribution of
pair coalescence times, $q(T_2 )$, via $p(\ell ) \sim \int dT_2 q(T_2 )
e^{-\rho \ell T_2} $ In neutrally evolving populations of constant size,
pair coalescent times are exponentially distributed with mean
$\mTtwo=N$. Consequently, the length of IBD segments is distributed as
$p(\ell) \sim 1/(1+\rho \ell \mTtwo)$ and has a long slowly decaying
tail.  If $N\sigblockT\gg 1$, coalescence is delayed since it happens predominantly after
lineages have reached the upper tail of the distribution of fitnesses
among different alleles of a linkage block. Hence the distribution of
pair coalescence times is peaked at $\mTtwo$ (now controlled by
selection and scaling with $\sigblockT^{-1}$).  Due to this delay of
coalescence, $p(\ell ) \sim e^{-\rho \ell \mTtwo}$ is approximately
exponential making long IBD segments much more unlikely than in the
neutral case -- at least in large populations.

%%%%%%%%%%%%%%%%%%%%%%%%%%%%%%%%%%%%%%%%%%%%%%%%%%%%%%%%%%%%%%%%%%%%%%%%%

\section{Discussion}
In most sexual populations, the histories of different chromosomes or
loci far apart on a chromosome are weakly correlated. Nearby loci,
however, are more tightly linked, which results in correlated histories
and linkage disequilibrium. Since the density of heterozygous sites
is $\pi = 2\mu \mTtwo$ and the length scale of LD is $\blockT = (\xo
\mTtwo)^{-1}$, the typical number of SNPs in one linkage block is on the
order of $n \sim \mu/\xo$. While this number may be small in obligately
outcrossing species, it can be large in regions of low recombination or
species that outcross rarely, including many plants, nematodes, yeasts,
and viruses
\cite{bomblies_local-scale_2010,barriere_high_2005,Neher:2010p32691,tsai_population_2008}.
If $n$ is much larger than one, and
a sizeable fraction of those affect fitness, different haplotypes
segregating within such a block will display a broad distribution in
local fitness with a variance that we have denoted by
$\sigblockT^2$. Neutral alleles linked to haplotypes drawn from this
distribution will be affected by linked selection, a.k.a.~genetic
draft. This in turn results in genealogies different from standard
neutral models but similar to the Bolthausen-Sznitman coalescent (BSC)
characteristic of rapidly adapting asexual populations
\cite{neher_genealogies_2012,Neher:2011p42539}. 
%Evidence for this
%scenario has recently been found in populations of Pristionchus
%pacificus (REF TO WORM PAPER).

The scenarios described in this paper are relevant whenever selection on
fitness differences between haplotypes of length $\blockT$ are larger
than the inverse population size, i.e., $N \sigblockT >1$.  The fitness
variance per block is given by $ \sigblockT^2 = \langle s^2\rangle \pi
\blockT$, where $\langle s^2\rangle$ is the second moment of the effect
distribution of polymorphisms. Hence
we require $N^2\langle s^2\rangle > (\pi \blockT)^{-1}$.  Provided
$n=\pi \blockT$ is large
enough, even nominally neutral ($Ns <1$) polymorphisms collectively
dominate the dynamics of haplotypes of length $\blockT$. This situation
is what we refer to as the ``genomic dark matter'' scenario.


\subsection*{Relation to previous work.}
Most previous work on genetic draft and selective interference
considered mutations with strong effects that behave deterministically
at high frequencies, while our focus here was on genomic dark matter.
Reduction of genetic diversity by sweeping beneficial mutations was first
discussed by Maynard Smith \cite{Smith:1974p34217}; see also
\cite{Barton:1998p28270,Gillespie:2000p28513}. In these models, genetic
diversity is determined by the typical waiting time between two
successive selective sweeps close enough to affect a given locus.
Similarly, deleterious mutations reduce diversity at linked sites. 
Assuming that mutations have a large detrimental effect on fitness and
happen with rate $\mu$ per site,
\cite{Hudson:1995p18197,Nordborg:1996p18149} have shown that the
reduction of genetic diversity is a function of $\mu/\xo$. As in our
analysis here, the strongest effect on genetic diversity comes from
tightly linked loci.

Weissman and Barton \cite{weissman_limits_2012} investigated the rate of adaptation and
its effect on diversity using scaling arguments similar to the one
presented here. In their model, adaptation is driven by individual
selective sweeps. The duration of a sweep explicitly sets the
time scale $\mTtwo$ on which coalescence happens. In this model, the
speed of adaptation is proportional to the maplength. In
contrast, our model assumes many weak effect mutations, and the
time scale of coalescence is set by $\sigblockT$, which is
self-consistently determined and itself depends on model parameters
such as $\xo$ and $\mu \langle s^2\rangle$. We can
recover their result for the rate of adaptation by
setting  $\mTtwo\sim s^{-1}$ and $\blockT
\sim s/\xo$. With these assumptions, we obtain
\begin{equation}
  \label{eq:weissman}
  \sigma^2 \sim L \xo s
\end{equation}
instead of \EQ{adaptation_diffusive}. The model used in 
Ref.~\cite{weissman_limits_2012} applies to a limit where at most one
strongly selected and sweeping mutation falls into one linkage block,
while our analysis considers the opposite limit. The basic properties of
genealogies and SFS are expected to be qualitatively
similar in the limit of one sweep per block. If the contribution from
weak mutations is negligible while sweeps are common, the coalescence
properties will be dominated by sweeps at different distances. This
limit has been studied in \cite{Durrett:2005p40919} and also results in
a multi-merger coalescent.

We have previously investigated adaptation and genetic diversity in
facultatively sexual populations that outcross rarely but recombine via
many crossovers in the event of outcrossing
\cite{Neher:2010p30641,Neher:2011p42539}; see also
\cite{weissman_limits_2012}. In this case, the entire genome remains
fully linked for times of the order of the inverse outcrossing rate, but
all loci are decoupled in each outcrossing event.  If the outcrossing
rate is much larger than fitness differences between individuals, the
dynamics at neutral loci is only weakly affected. However, as soon as
the outcrossing rate becomes comparable to $\sigma$, the diversity is
strongly reduced, genetic draft becomes strong and genealogies resemble
the BSC \cite{Neher:2011p42539}. This type of model is appropriate when
the outcrossing rate is comparable to fitness differences between individuals and many loci
segregate independently in these rare recombination events. Examples of
such behavior include viruses like influenza or HIV. If outcrossing is
frequent relative to $\sigma$ and the average number of crossovers is
small, interference is dominated by tightly linked polymorphisms and the
analysis presented here applies.



\subsection*{Genomic dark matter versus sweeps.}
In the limit of small effect mutations -- the genomic dark matter --
beneficial and deleterious mutations have similar impact on diversity
and coalescence since all that matters is their contribution to fitness
variation. Inferences about their effect sizes are impossible.

However, one often is interested in identifying particular mutations
that arose in response to specific environmental challenges. If
successful, those mutations tend to be of large effect and fall outside
the scope of our model. Our model, however, provides the background on
top of which such singular adaptations can be sought and understanding
the statistical patterns of diversity and linkage within this null model
is essential for reliable inference. Importantly, strong 
adaptations only perturb a fraction of the genome (more precisely a
segment of length $\approx s(\xo \log Ns)^{-1}$, where $s$ is the
selection coefficient). 

The genomic dark matter might also be responsible for the ``missing
heritability'' in many genome wide association studies. If much of the
phenotypic variation is due to a large number of loci with small
effects, association studies will not be able to pinpoint the causal
mutations, even though a trait is highly heritable \cite{Yang:2010p35988}.

%%%%%%%%%%%%%%%%%%%%%%%%%%%%%%%%%%%%%%%%%%%%%%%%%%%%%%%%%%%%%%%%%%%%%%%%%
\section{Model and Methods}
We use a model with discrete generations, haploid individuals, an approximately constant
population size and a finite number of sites in the genome, as
implemented in FFPopSim \cite{zanini_ffpopsim:_2012}. We simulate a
fraction of a chromosome of length $L$, where recombination happens with
rate $\xo$ between randomly chosen gametes. If $\xo L\ll 1$, no
recombination happens in most cases. In addition to forward simulation,
we also track the genealogy of a central locus which allows us to measure pair
coalesence times, the time to the MRCA, and the neutral SFS directly
(this functionality is implemented in a more recent release of FFPopSim;
see \url{http://code.google.com/p/ffpopsim}). For all parameters, we
produce equilibrated populations by simulating for 10
$T_{MRCA}$. Subsequent measurements of population parameters start from
these equilibrated populations and sample the population roughly twice
every $\mTtwo$ as estimated from our theoretical arguments. All scripts
associated with this paper can be downloaded from
\url{http://git.tuebingen.mpg.de/reccoal}.

\subsection*{Fixed fitness variance.~}
For simulations at fixed fitness, we specify $\sigma^2$ and, each generation, rescale
the selection coefficients associated with individual loci accordingly.

Mutations are introduced into a random individual whenever a locus
turns monomorphic, i.e., the previously introduced mutation was lost
or has fixed (see \cite{Neher:2011p42539}). This allows us to simulate
a large number of sites efficiently in a limit where the overall
mutation rate is small compared to $\mTtwo$. In this way, we keep all $L$ loci
polymorphic without employing a high mutation rate, which would result
in frequent recurrent mutations.

We simulate a grid of parameters with $N$ taking the values $[1000,
3000, 10000]$, $\sigma$ the values $[0.01, 0.03, 0.1]$, and $\ox$ five
logarithmically spaced values between $0.1\sigma$ and $1.0 \sigma$. For the
analysis, simulations were filtered so that $\blockT>30$ and
$\blockT<L/3$. To prevent invalid logarithms, $\log(N\sigblockT)$ was
replaced by $\log(N\sigblockT+1)$ in \EQ{self_consist}.


\subsection*{Adaptation.~}
Instead of fixing the variance, we now fix the selection
coefficients. The variance is hence a fluctuating but roughly constant
quantity. Again, mutations are introduced into a random individual
whenever a locus turns monomorphic. 
The grid of parameters in this case was $L\in [3000, 10000]$, $N\in
[1000,3000, 10000]$ and $s \in [0.001, 0.003, 0.01]$. For the
analysis, simulations were filtered such that $\blockT>30$ and
$\blockT<L/3$. 





\subsection*{Dynamic Balance.~}
In this set of simulations, we simulate a genome consisting of finite sites in a
constant fitness landscape where mutations at each locus have a small
effect $s$. Mutations are injected at random with rate $\mu$ at each
locus. In contrast to the models above, where mutations are injected
only when a locus is monomorphic, in this
case we allow recurrent and reversing mutations, which make possible the dynamic balance state. 

The grid of parameters in this case was $L\in [3000, 10000]$, $N\in
[1000,3000, 10000]$, $s \in [-0.001, -0.003, -0.01]$, $L\mu \in [1,3,
10, 30]$, and $\xo$ logarithmically spaced between $s$ and $1.0$. 
For the analysis, simulations were filtered such that $\blockT>30$, 
$\blockT<L/3$, and $\mTtwo \mu <0.5$.


%%%%%%%%%%%%%%%%%%%%%%%%%%%%%%%%%%%%%%%%%%%%%%%%%%%%%%%%%%%%%%%%%%%%%%%%%
\section{Acknowledgements}
We would like to thank Fabio Zanini for stimulating discussions and help
with FFPopSim. This work is supported by the ERC starting grant HIVEVO
260686 to R.A.N and in part by the NSF PHY11-25915 grant to
KITP. B.I.S. acknowledges support from NIH R01 GM086793.
%%%%%%%%%%%%%%%%%%%%%%%%%%%%%%%%%%%%%%%%%%%%%%%%%%%%%%%%%%%%%%%%%%%%%%%%%

%%%%%%%%%%%%%%%%%%%%%%%%%%%%%%%%%%%%%%%%%%%%%%%%%%%%%%%%%%%%%%%%%%%%%%%%%
%\bibliographystyle{natbib}
\bibliography{/ebio/ag-neher/share/bibliography/bib}
%\newpage
%\appendix
%\clearpage
%\onecolumngrid
%\setcounter{figure}{0}
%
%%% Local Variables: 
%%% mode: latex
%%% TeX-master: t
%%% End: 

\section*{Supplementary figures: dark genetic matter}

\begin{figure}[h]
  \centering
  \includegraphics[width=0.38\columnwidth]{T2_dynbalance}
  \includegraphics[width=0.38\columnwidth]{SFS_dynbalance}
  \includegraphics[width=0.38\columnwidth]{LD_dynbalance}
  \caption{Coalescent time, site frequency spectra and LD for the model with many recurrent mutations of weak effect.}
  \label{fig:dark_matter_suppfig}
\end{figure}

\clearpage
\section*{Supplementary figures: beneficial mutations with fixed effect}

\begin{figure}[h]
  \centering
  \includegraphics[width=0.38\columnwidth]{T2_fixed_effect}
  \includegraphics[width=0.38\columnwidth]{SFS_fixed_effect}
  \includegraphics[width=0.38\columnwidth]{LD_fixed_effect}
  \caption{Coalescent time, site frequency spectra and LD for the model
    with beneficial mutations with fixed effect.}
  \label{fig:dark_matter_suppfig}
\end{figure}


